Respuestas (numéricas) a los
ejercicios propuestos
Si encuentran discrepancias, ¡¡¡CONSULTEN!!!, porque
quizás haya algún resultado erróneo que hayamos pasado por
alto.
Recuerden que es la versión 2.0 de la materia, así que tenemos
que ir afinando el lápiz y corrigiendo asperezas para cursos
futuros. Ayúdennos a lograrlo 
Práctica N°1. CINEMÁTICA.
Ej. 1: I) B = 26.6 m, 208.9°
Ej. 2: I) 57°; II) Cx = -2.2, Cy = 4.5 ó Cx = 2.2, Cy = -4.5
Considerando que g tiene el valor numérico: 10 m/s2
Ej. 3: 1.2 km
Ej. 4: -4m/s, 0.4 s
Ej. 5: I) si considero y+ para arriba con el origen en el piso obtengo los siguientes
resultados; II) 2 s: 40 m y 10 m/s, 4 s: 40 m y -10 m/s, 8 s: -80 m y -50 m/s; los
desplazamientos son, 0-2 s: 40 m, 2-4 s: 0 m, 4-8 s: -120 m; III) 6 s; VI) 3 s y 45 m; V) 1 s y 5
s; VI) -
Ej. 6: I) 40 m/s; II) 180 m; III) -
Ej. 7: I) 2s: 40 m, 5 s: 25 m, 8 s: -80 m; II) 2 s: vx = 40 m/s, vy = 10 m/s, 5 s: vx = 40 m/s, vy =
-20 m/s, 8 s: vx = 40 m/s, vy = -50 m/s, cuando habla de cuatro posiciones conocidas, se
refiere al instante t = 0 s (vx = 40 m/s, vy = 30 m/s), además de los otros tres instantes ya
mencionados ; III) 6 s, vx = 40 m/s, vy = -30 m/s; VI) 3 s, vx = 40 m/s, vy = 0 m/s; V) y = 3/4x-
x2/320 (tomando la colina desde donde parte el viejito de Newton, como y = 0 en mi sistema
de referencia)
Ej. 8: I) 4.4 m; II) 8 m/s; III) y = -5/36x2 (tomando el piso como y = 0 en mi sistema de
referencia)
Considerando que g tiene el valor numérico: 9.8 m/s2
Ej. 9: I) 10 m/s; II) -4.68 m/s
Ej. 10: I) 18.76 m/s, 31.5°; II) r = 30.4 i + 0.93 j m, v = 16 i + (-8.82) j m/s
Ej. 11: I) 1.15 s; II) 12 m; III) antes de haber llegado a la altura máxima
Ej. 12: I) H = v0 sin2θ0/2g; II) R = v0 sin(2θ0)/g; III) 53.1°
Ej. 13: II) θ0 = π/4 + φ/2, d (máxima) = v02/[g (1+sinφ)]
Ej. 14: I) 9 x 1022 m/s2; II) 1.52 x 10-16 s
Ej. 15: I) 33.6 min; II) 27.8 min
Práctica N°2(A). ESTÁTICA.
Ej. 1: TA = 166.2 N, TB = 57.3 N, TC = 125.21 N
Ej. 2: I) 33.9 N; II) 39.2 N
Ej. 3: I) 0.735 m/s2; II) m1 hacia arriba por el plano y m2 hacia abajo; III) 20.8 N
Ej. 4: I) T1,2,3 = Mg/2, T4 = 3Mg/2, T5 = Mg; II) F = T1 = Mg/2
Ej. 5: 103 NPráctica N°2(B). DINÁMICA.
Ej. 1: Fx = 2.5 N, Fy = 5 N, F = 5.6 N
Ej. 2: I) a = 1 m/s2 i + (-1.2 m/s2) j; II) a = 1.6 m/s2, -50° (medido desde el suelo)
Considerando que g tiene el valor numérico: 10 m/s2
Ej. 3: a = -0.357 m/s2 (suponiendo y > 0 hacia la derecha en los diagramas de cuerpo libre),
T = 37.14 N
Ej. 4: I) a = 2 m/s2 (considerando y > 0 hacia arriba, además téngase en cuenta que y = 0 lo
estamos poniendo en el piso); II) m2 = 1 kg; III) N3 = 36 N
Ej. 5: desciende acelerando
Ej. 6: I) a = 2.5 m/s2 (suponiendo y > 0 hacia la derecha y abajo en los diagramas de cuerpo
libre); II) a = 7.5 m/s2
Ej. 7: I) a = 2.94 m/s2 (suponiendo y > 0 hacia la derecha en los diagramas de cuerpo libre);
II) se detuvo, ergo no hay fuerza de rozamiento; III) FMIN = 24.5 N
Ej. 8: I) d = 50 m (suponiendo y > 0 hacia la derecha en los diagramas de cuerpo libre); II)
25 m
Ej. 9: I) a = 2.66 m/s2 (suponiendo y > 0 hacia la derecha y abajo en los diagramas de
cuerpo libre, el sistema gana velocidad); II) a = 4 m/s2 (el sistema pierde velocidad)
Ej. 10: I) FMAX = 4.286 N (suponiendo y > 0 hacia la derecha y abajo en los diagramas de
cuerpo libre); II) m3:MAX = 4 kg
Ej. 11: I) 10 cm; II) 10 cm; III) a = 5 m/s2
Considerando que g tiene el valor numérico: 9.8 m/s2
Ej. 12: I) 4.9 m/s2; II) 2 m/s2 hacia arriba, T = 118 N
Ej. 13: 488 N
Ej. 14: I) 1.78 m/s2; II) 0.368; III) 9.4 N; IV) 2.67 m/s
Ej. 15: I) la fuerza de fricción estática; II) 34.7 N; III) 0.306
Ej. 16: v2 = Mgr/m
Ej. 17: I) la fuerza de fricción estática; II) 0.085
Ej. 18: I) Fhorizontal ≤ 2mg tg(α) con α el ángulo de apertura de las secciones triangulares, a = g
tg(α), FA sobre B, B sobre A = mg/cos(α); II) F ≤ 2 m g [sen(α) + μsup cos(α)]/[cos(α) − μsup sen(α)]; III)
F ≤ 2mg [tg(α)+ μsuelo]
Práctica N°3. TRABAJO Y ENERGÍA.
Considerando que g tiene el valor numérico: 10 m/s2
Ej. 1: F = 25 N (suponiendo y > 0 hacia la derecha)
Ej. 2: I) 3.6 x 106 J; II) 36 km (una muestra de lo importante que es no desperdiciar
energía)
Ej. 3: W = ∆Ecin = 150 J, W = ∆Ecin = 0 J
Ej. 4: 3.46 m/s (suponiendo y > 0 hacia la derecha)
Ej. 5: caída libre: 10 m/s, plano inclinado: 10 m/s, Cerro Otto: 10 m/s
Ej. 6: 3.16 m/s (suponiendo y > 0 hacia la derecha)
Ej. 7: d = 0.4 m
Ej. 8: I) K = 480 N/m; II) Wroz= 41 J, Wela = -60 J
Considerando que g tiene el valor numérico: 9.8 m/s2
Ej. 9: W = -6 J
Ej. 10: I) “k” = 575 N/m; II) W = 46 J
Ej. 11: I) Wgrav = -168 J; II) Wfric = -184 J; III) Wtension = 500 J; IV) ∆Ecin = 148 J; V) v = 5.64
m/s
Ej. 12: 9.800.000
Ej. 13: I) v = 5 m/s; II) 13° por encima de la horizontal; III) 64 J
Ej. 14: I) h = 3/4 r; II) d = 0.1676 r; III) h = 2/3 r; IV) v0 = √(gr)
Ej. 15: I) U = -9.42 x 109 J; II) Emec = -4.71 x 109 J
Ej. 16: I) VA = 8.02 m/s; II) VB = 6.12 m/s; III) T = 527.8 N; IV) Pfric = -143.6 W; V) V0 = 5.60
m/sEj. 17: I) H = 2/3 R; II) v(θ)=√{2gR[1-cos(θ)]}; III) θ = 48.19°
Ej. 18. I) N = 49 kN; II) R = 16 m
Práctica N°4. DINÁMICA DE UN SISTEMA DE
PARTÍCULAS.
Ej. 1: 9x105 J
Ej. 2: Ecpelota > Ecbola
Ej. 3: p2 = p1 sin(θ3)/sin(θ2+θ3). Resultado análogo para p3
Ej. 4: y = d/16
Ej. 5: v2,0 = 3/2 v1,0 ; v2,f = 3/√2 v1,0
Ej. 6: I) vf = 37.5 km/h; II) 1/4 Ecinicial
Ej. 7: I) h = L[1-cos(α)]; III) v1 = √(2gh)(1+M/m)
Ej. 8: yCM = 950 m (sobre la base)
Ej. 9: 87.5 N
Ej. 10: 4M √(gl)/m
Ej. 11: h = 0.556 m
Ej. 12: vCM = [1.4 i + 2.4 j] m/s; PCM = [7 i + 12 j] kg m/s
Ej. 13: (xCM, yCM) = (π/4, π/4)
Ej. 14: (xCM, yCM) = (12/25, 3/7)
Práctica N°5. DINÁMICA DE UN CUERPO RÍGIDO.
Considerando que g tiene el valor numérico: 9.8 m/s2
Ej. 1: I) Ix = 4mb2, Iy = 4ma2; II) Ix’ = 8mb2, Iy’ = 8ma2; III) Iz’ = 8m(b2+a2)
Ej. 2: Iz = 247/480 MR2
Ej. 3: 2/3 MR2
Ej. 4: Ix = Iy = Iz = 2/5 Msemiesfera R2
Ej. 5: I = 1/3ml2 + Ml2 + 1/2MR2
Ej. 6: W = 63 J
Ej. 7: γ = 200/s2
Ej. 8: 0.357 m
Ej. 9: I) 120 J; II) T1 = 35.28 N, T2 = 32.34 N
Ej. 10: ICM = 16.18 kg m2
Ej. 11: 0.291 m/s
Ej. 12: e = 0.06 m; aCM = 15.1 m/s2; γ = 75.5/s2
Ej. 13: F1/F2 = 2 √3/3 (h/d)
Ej. 14: II) a = [1/(1+ICM/m2R2)]g; III) T = [ICM/(ICM+m2R2)]m2g; IV) v = √[2gh/(1+ICM/m2R2)]
Ej. 15: Fv = 0.07 N (hacia abajo), Fh = 0.7 N (hacia la izquierda)
Ej. 16: τ = 0.121 Nm, aCM = 0.228 m/s2
Ej. 17: I)II) wf = 2/s, ∆Ec = 200 J; III) wf = 1/s, ∆Ec = 0
Práctica N°6. MOVIMIENTO OSCILATORIO.
Ej. 1: I) 48π2 N/m; II) 0.24π2 J; III) x(t) = 0.1 m × sen(4π t × 1/s)
Ej. 2: I) Ecin/Etotal = 3/4; II) √2/2 A
Ej. 3: μestatico = 0.37
Ej. 4: T = 1.4 s
Ej. 5: T = 2π√(l/g)
Ej. 6: I) t = π√(m/k); II) A = v0√(m/k)Ej. 7: I) x(t) = 0.15 sin(8πt) m, v(t) = 0.15 (8π) cos(8πt) m/s, a(t) = -0.15 (8π)2 sen(8πt) m/
s2, Ec(t) = 2,84 cos2(8πt) J, Ep(t) = 2,84 sen2(8πt) J, EM(t) = 2.84 J; II) Damos algunos
resultados, si los corroboran, entonces la mecánica usada es la correcta y podrán resolver
el resto sin inconvenientes: xmax = 0.15 m, tmax = 1/16 s, xmin = 0 m, tmin = 0 s; vmax = 3.77 m/s
cuando x = 0 m, t = 0 s, vmin = 0 m/s cuando x = 0.15 m, t = 1/16 s; y así con la aceleración y
las energías cinética, potencial elástica y mecánica (total).
Práctica N°7. ELASTICIDAD.
Considerando que g tiene el valor numérico: 9.8 m/s2
Ej. 1: II) 2 kg
Ej. 2: II) T = 74. 4 N, Rx = 57 N, Ry = 50.2 N
Ej. 3: II) 0.9 m
Ej. 4: ∛(Mg/YA)
Ej. 5: 0.588 × 10-3 por cada uno de los dos pares de fuerzas cortantes
Ej. 6: ∆p = 1.05 × 105 N/m2.
Ej. 7: I) ∆V = -2.8 × 10-8 m3; II) 3.27 × 10-5.
Ej. 8: T1 = 5081.5 N, T2 = 14517.5 N
Ej. 9: R1 = (l2/L) W, R2 = (l1/L) W
Ej. 10: I) a = - 0.4F/(ρLA) (hacia la izquierda); II) ∆L1 = 3.05 FL/(YA), ∆L2 = 9.8 FL/(YA),
∆L3 = 15.2 FL/(YA), ∆Ltotal = 28.05 FL/(YA)
Ej. 11: ∆H = FH/(2πR2Y)
Práctica N°8. ONDAS.
Ej. 1: I) y1 e y2 son las únicas que se pueden escribir como f(x ± vt); II) v1 = 2 m/s (hacia la
derecha), v3 = -1/2 m/s (hacia la izquierda); III) y1.
Ej. 2: I) v = 480 cm/s; II) y(x,t) = 4 sen[2π(x/24 - 20t)] en el sistema CGS.
Ej. 3: I) y(0,0) = 0, v(0,0) = -8π, en el extremo de la cuerda donde se encuentra el vibrador
(x = 0), cuando se lo enciende (t = 0) realiza un movimiento armónico inicialmente dirigido
hacia abajo (v(0,0) < 0), mientras que la onda se propaga hacia la derecha; II) A = 10 cm, v =
0.5 cm/s, λ = 1.25 cm; III) t = 20 s, y(10,t) = 10 sen[2π(8 - 0.4t)], v(10,t) = -8π cos[2π(8 -
0.4t)], a(10,t) = -6.4π2 sen[2π(8 - 0.4t)].
Ej. 4: I) 37.4 × 10-3 m; II) vlong = 5.06 × 103 m/s, vtrans = 490 m/s.
Ej. 5: I) t = d/v; II) t = 4d/v; III) t = 2d/v.
Ej. 6: I) v = 48 m/s; II) vmax = 4.8π m/s, amax = 1152π2 m/s2; III) A = 0.028 m; IV) 0.037 m.
Ej. 7: I) λ = 12.56 m, f = 6.37 Hz, v = 80 m/s; II) y(x,t) = 0.030 cos(x/2)cos(40t); III) nodos:
x = ±(2n+1)π, antinodos: x = ±2nπ; IV) A = 2.94 cm.
Ej. 8: I) 4to y 5to; II) T1/T2 = 16/25; III) f = 350 Hz; IV) m = 400 kg.
Ej. 9: II) 0.71 × 10-8 m.
Ej. 10: I) f = 2040 Hz; II) a) ondas estacionarias, b y c) interferencia constructiva.
Ej. 11: I) f = 916.7 Hz; II) 858.3 Hz; III) 806.9 Hz; IV) 733.6 Hz.
Ej. 12: I) f = 25 Hz, T = 0.04 s, λ = 5 m, v = 125 m/s; II) vy(x,t) = 2.5π cos[(0.4π)x + (50π)t]
m/s; III) 0.10 cos[(0.4π)x] sin[(50π)t] m/s; IV) 2.5 m.
Ej 13: I) f = 10 Hz, v = 100 m/s, μ = 5 × 10-3 kg/m; II) 8 sin[(0.1π)x] sin[(10π)t] cm/s; III)
vymax = 80π cm/s.
Ej. 14: T = 50.26 N.
Ej. 15: I) f = 158.1 Hz; II) n = 7.
Ej. 16: 1.7 m/s.Práctica N°9. HIDROSTÁTICA E
HIDRODINÁMICA.
Ej. 1: I) F = 407070 N; II) h = 3.34 m.
Ej. 2: I) F = 1111 N (10 veces menos que el peso del auto); II) h = 0.45 M.
Ej. 3: I) F = 481.06 N; II) a = 29.4 m/s2.
Ej. 4: v = √(2gh), t = 2√h (A1/A2) √{[1-(A2/A1)2]/(2g)}.
Ej. 5: I) 5 m/s; II) h = 1.28 m; III) ∆p = 37500 Pa; IV) ∆y = 0.28 m.
Ej. 6: P = 3.8 × 102 kPa, v = 3.9 m/s
Ej. 7: I) sin α = (h/L)√(ρ0/ρ1); II) hmin = L√(ρ1/ρ0) ; III) T = Mg[√(ρ0/ρ1)-1] con “M” la masa de
la varilla.
Ej. 8: I) Q = 0.4 m3/s; II) 0.02 m/s; III) P = 2.418 × 102 kPa; IV) P = 2.326 Pa
Práctica N°10. CALOR Y TERMODINÁMICA.
Ej. 1: 10340.33°F, 37°C, 73.89°C, -297.4°C.
Ej. 2: T = 472.83°C.
Ej. 3: β1 = [P2-P1+(P-P1)β(t2-t1)]/[(P-P2)(t2-t1)].
Ej. 4: I) tf = 25.76°C; II) no depende de m.
Ej. 5: I) ∆U = 12500 cal; II) Qadb = 15000 cal (absorbido); III) Qba = -7500 cal (libera); IV) Qad
= 12500 cal (absorbido), Qdb = 2500 cal (absorbido).
Ej. 6: 4.11 × 10-6 mol
Ej. 7: demostración
Ej. 8: 66.1 J, 65.2 J
Ej. 9: W = [1/(γ-1)]V[p1-p2(p1/p2)1/γ]
Ej. 10: I) (B) p = 25 kPa, V = 0.004 m3, T = 800 °K, (C) p = 25 kPa, V = 0.002 m3, T = 400 °K;
II) III) IV) (A-B) Q = 690.4 J, W = 690.4 J, ∆U = 0 J, (B-C) Q = -1744.88 J, W = -500 J, ∆U =
-1244.88 J, (C-A) Q = 1244.88 J, W = 0 J, ∆U = 1244.88 J, V) Wneto = 190.4 J
Ej. 11: WIAF = 405.32 J, WIF = 304 J, WIBF = 202.7 J
Ej. 12: II) Wneto = p1(V2-V1)ln(p2/p1)